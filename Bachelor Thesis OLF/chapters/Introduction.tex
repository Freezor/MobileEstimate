\chapter{Introduction}

%Thema
Most of the contracts IT companies subscribe are projects and these are notorious for going past their deadline and over their budget. According to the study of Capgemini in 2014 \cite{capgemini}, the importance of cost estimation increases every year. The study asked for the most important requirements in the IT for the next years. Top requirement, as asked in the study, is to increase the efficiency, which means to lower the costs and to meet determined deadlines. This will increase the effort companies have to take in planning their projects.\\
%Fokus / Aspekt
All businesses also want to lower the risk of delayed or canceled projects. This means IT companies have to take more effort in requirements engineering and cost estimation to give their clients an accurate estimation of the upcoming project. This results in a increasing effort for this in IT projects. As cost estimation is a part of requirements engineering, it will lower the chance of a budget overrun. To achieve this, 6\% to 12\% of the project time has to be spend in requirements engineering. The budget overrun is limited with this spent time to a maximum of 50\% of the estimated cost\cite{Partsch}. Which means, that a higher effort in requirements engineering and cost estimation will lower the chances of failed projects, but to much effort has not more impact.\\
Therefore cost estimation is an important element for planning software projects and can be responsible for successful or failed projects. If a project was estimated right, budget overrun will be noticed early an retaliatory action can be executed. It is even more important to estimate as precisely as possible to guide the project to success. There are several methods for these estimations that can be used at different phases of the project. These methods of estimation lean their result on the information they get from the development process and the artifacts of the particular project phase. They include \textit{requirement documents}, \textit{diagrams} or the \textit{program code} itself. All available artifacts depend on the used process model and the project phase \cite{EntwicklungKompakt}. Based on the described information, the actual project can be categorized, so that the \textit{'best fitting'} estimation technique for the current estimation can be selected. These methods can be time-consuming and related projects can often only be found in the own company context or are based on experiences.\\
%Methode
This paper aims to develop a mobile application which supports the Function Point estimation. The application aims to make the estimation process in IT projects simpler and more efficient. To achieve a good cost estimation experience on smartphones, the most important design guideline was \textit{'Only show what I need when I need it'} \cite{materialdesign}. The comparison between projects has to be \textit{formalized} and \textit{implemented} to give the user an overview of terminated projects and how they have been estimated. He should have the possibility to transfer such an estimation to a new project or get a quick view how much man days it took.\\
%Ergebnisse
The developed application \textit{MobileEstimate} should prove that cost estimation on mobile devices is possible. It should allow to estimate the costs of a project with the \textit{Function Point} method and among the existing projects it should display \textit{related projects}. Estimation results from a related project are possible to be viewed in the application and \textit{transfered} to another estimation.\\

