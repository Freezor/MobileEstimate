%\kurzfassung

%\newcommand{\kurzfassung}[1][Abstract]{\chapter*{#1}\markboth{#1}{#1}}
%\kurzfassung
%\newpage

\chapter*{Abstract}

%\begin{abstract}
%% deutsch
\paragraph*{}
Die Wichtigkeit von Kostenschätzungen von IT-Projekten steigt durch immer komplexere und größere Projekte stetig an. Schwierigkeiten bei der Kostenschätzung liegen im Zugriff auf bereits vorhandene Schätzungen, die zur Verbesserung der Schätzwerte beitragen, sowie der Auswahl relevanter Projekte, damit nicht alle verfügbaren betrachtet werden müssen. Diese Arbeit zielt darauf ab, den Prozess der Kostenschätzung als mobile Anwendung umzusetzen und die genannten Probleme mit einem dynamischen Prozess durchzuführen. Hierbei steht die Umsetzung des Function Point-Verfahrens im Fokus, sowie die Möglichkeit mit einem Vergleich von abgeschlossenen Projekten auf deren Schätzung zuzugreifen. Diese mobile Anwendung wurde als Android Applikation umgesetzt und trägt den Namen \textit{MobileEstimate}. Es ergeben sich dadurch neue Möglichkeiten für Projektmanager und Projektteams zur schnellen und einfachen Kostenanalyse bei IT Projekten sowie zum Monitoring der Projektkosten.

%% englisch
\paragraph*{}
The importance of cost estimation in IT projects is constantly increasing through more complex and larger projects. Some of the difficulties in cost estimations are the access to existing estimations, which help to improve the estimation, and the selection of relevant projects in order to not consider all available projects. This paper aims to implement the cost estimation as a mobile application with a dynamic process to approach the specified problems. Implementation of the Function Point method and the possibility to compare projects and get access to their cost estimation is the focus of this paper. The mobile application was implemented as an Android application and is called \textit{MobileEstimate}. This results in new opportunities for project managers and project teams for fast and easy cost analysis for IT projects and the monitoring of project costs.

%\end{abstract}