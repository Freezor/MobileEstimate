\chapter{Conclusion}\label{conclusion}

This paper presented the possibility to process \textit{cost estimation} with a mobile application. It tries to give an approach to make projects \textit{comparable} and build a base for a mobile cost estimation application. It focuses on the implementation of \textit{Function Point}. This estimation technique is completely workable. The estimation can also be exported as an excel sheet for further processing. The project was estimated as \textbf{14} \textit{man days}. The Development itself was done in \textbf{19} \textit{man days} which is a difference of the estimated time of \textbf{5} \textit{man days}.\\
The developed application allows a \textit{comparison} between projects which relies on \textit{seven} different project properties. These are calculated with a distance which gives the \textit{total difference} of two projects. Those project properties are implemented and allow a categorization and give traceable relation results. To use the application properly it is assumed that the user knows how the \textit{estimation techniques} work. He has the possibility to use the \textit{Help} inside the application but a basic understanding is required.\\
One of the biggest problems was to create a user interface which inherits all functions to estimate a project and also supports the \textit{android design principles}. For the creation of the user interface many \textit{ListViews} are used, which allow the possibility for quick and dynamic extension of these lists. These caused an extra effort for making them save all informations, but afterwards the use of \textit{ListViews} is unproblematic. Another problem was to create a \textit{algorithm} for comparing projects. It had to be evaluated which properties make sense and can be set for every project. These properties must also have relevance for the algorithm. To create this algorithm I had to develop a possibility to calculate a distance which can be transfered into a percentage difference. \\
As an outlook for further development many features and additions are possible. One addition is to implement more estimation techniques such as \textit{COCOMO}. This would support more informations about how long a project can take because different technique have another effort as output. It has to be evaluated if as a consequent, it is needed to make a project transferable to other estimation techniques or if projects need a different structure which allows more estimation techniques. Another addition is to create a \textit{server} which stores all estimations from the user. This would support more possibilities to improve the estimations and all users with access could rely on the informations of other users. This would constantly expand the estimations and would give more precise results. It could also allow sharing of projects which gives other users the possibility to look up estimations in detail and adapt those informations for their project. A server management tool would also provide the possibility to manage the \textit{master data} of the project comparison from the server. This would make it easy to create more options for the project properties. An \textit{adaption} of the application for tablets would provide the possibility of a different user experience.\\
All in all it can be said, that the mobile application for cost estimations offers a \textit{promising alternative} to the traditional estimation software, that is worth further \textit{research} and \textit{evaluation}. The application will be extended within the \textit{'Projektstudium'} for my \textit{master degree} at the \textit{University of Applied Science in Trier}. In the winter semester \textit{2016/2017} this application will also be used in the module \textit{'Spezifikation interaktiver Systeme'} which will give more feedback about this application and allows a field study.