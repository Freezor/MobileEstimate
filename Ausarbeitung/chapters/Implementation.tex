\chapter{Implementation}

The used libraries and selected implementation approaches are described in this chapter. \textit{Android Studio} was used as the \textit{IDE} for development and \textit{GitHub} was used for version management. 

\section{Used Libraries}

For the implementation were two external libraries necessary. These libraries are called \textit{POI} and \textit{mpandroidchartlibrary}. \textit{POI} is a library from the \textit{APACHE Software Foundation} and is open source. It allows a conversion of data into Microsoft documents. This library is used
in the \textit{Export} component and allows to create \texttt{xls} documents with the project data. The \textit{mpandroidchartlibrary} library is also an open source library developed by Philipp Jahoda. This library provides many different diagram types for Android such as pie and bar charts. This charts are used for the estimation technique analysis and for the project statistics. 

\section{Project Structure}

Android divides the project into an \textit{assets}, \textit{resources} and \textit{java} folder. The \textit{assets} folder stores project icons and the two databases of the project. After installation of the \texttt{apk}, these files are moved to the internal folder space of the application. The \textit{resources} folder contains every \texttt{layout} file, \texttt{string} resources and the android \texttt{icons}. All resource files ar typically \texttt{XML} files or in case of images \texttt{png} files. The \textit{java} folder contains the complete source code. This folder was divided into the sub folders \textit{Activities}, \textit{Fragments}, \textit{DataObjects}, \textit{Server} and \textit{Util}. The folders \textit{Activities} and \textit{Fragments} contain the associated classes for each layout and are responsible to control the user input. In the folder \textit{DataObject} are all data classes stored. These are objects that only store data and don't have any logical components. The \textit{Server} folder contains at the moment only the \texttt{ServerConnector} class. This folder was created to separate the future server connection. Last folder is the \textit{Util} folder. It contains classes that process data, for example the \texttt{DatabaseHelper} class.

\section{Pre-filled Database}\label{exDB}

The advantage of a pre-filled database is a shorter initialization after first start of the application. The database doesn't have to run many \textit{SQL} scripts to create tables and insert values. Also values and tables can be created with an external tool such as \textit{SqliteBrowser}\footnote{\url{http://sqlitebrowser.org/}}. This allows testing of the tables and \textit{queries} that are used in the application. To use this database two steps are essential:
\begin{enumerate}
	\item \textbf{Create Database}
	\item \textbf{Open Database}
\end{enumerate}
Both methods are inherited in the class \texttt{DataBaseHelper}. Listing \ref{createdb} shows the method \texttt{createDataBase()} which is responsible for the creation of the database. At first the method checks with \texttt{checkDataBase()} if the database does already exist in the \texttt{application} folder. If it does not exist the method calls \texttt{copyDatabase()} which will read the database from the \texttt{assets} folder and write a new database to the application folder.
\begin{lstlisting} [caption=Method \textit{createDataBase}, label=createdb] 
public void createDataBase() throws IOException
{
	boolean dbExist = checkDataBase();
	if (!dbExist)
	{
		this.getReadableDatabase();
		try
		{
			copyDataBase();
		} catch (IOException e)
		{
			Log.d("ERROR", "Database could not be copied " + e.getCause());
			throw new Error("Database could not be copied");
		}
	}
}
\end{lstlisting}
Open the database is done with the method \texttt{openDataBase()}, which is described in listing \ref{opendb}. The class variable \texttt{projectEstimationDataBase} is initialized here with the \texttt{SQLiteDatabase} command \texttt{openDatabase} which create a \texttt{SQLiteDatabase} object with the path where it is stored. This allows the execution of \texttt{SQL} queries in this object.
\begin{lstlisting} [caption=Method \textit{openDataBase}, label=opendb] 
public void openDataBase() throws SQLException
{
	String myPath = DB_PATH + DB_NAME;
	projectEstimationDataBase = SQLiteDatabase.openDatabase(myPath, null,
										 SQLiteDatabase.OPEN_READWRITE);
}
\end{lstlisting}
\section{The Database Helper}

The \texttt{DataBaseHelper} class has the most lines of code in the project. As \texttt{SQL} queries can be generalized it is not possible to generalize the preprocessing of the data. Different date needs different treatment for the calling method. Each table has also various column names what is another difficulty for generalization. \\
All requests with \texttt{SELECT} statements have a similar structure. As described in listing \ref{selectStatement}, the first step is to build a \texttt{String query} with the table name and parameter from the method. Next step is to get a readable database from the \texttt{SQLiteDatabase} object of this class. The query will be executed on this object and creates a \texttt{Cursor}, which is the result table. To get the values from this \texttt{Cursor} each column has to be addressed and saved into a variable which will be returned to the caller. If more values are needed those have to be packed into other data structures.
\begin{lstlisting} [caption=Method \textit{openDataBase}, label=selectStatement] 
public void selectSomethingFromTable(String id)
{
	String query = String.format("SELECT _id FROM <TABLENAME> WHERE id = '%s'",
											 id);
	SQLiteDatabase db = this.getReadableDatabase();
	String name = "";
	try (Cursor c = db.rawQuery(query, null))
	{
		if (c.moveToFirst())
		{
			name = c.getInt(c.getColumnIndex("<COLUMN-NAME>"));
		}
	}
	db.close();
	return id;
}
\end{lstlisting}
For the query types \texttt{DELETE} and \texttt{ALTER TABLE} a method needs to instantiate a writable database with \texttt{'SQLiteDatabase db = this.getWritableDatabase()'}. The \texttt{db} objects offers the possibility \texttt{update} to change existing values in a table. It gets an object \texttt{ContentValues} as a parameter which inherits the column name and value which has to be updated. The other to parameters for \texttt{update} are the table name and selection to find the right row. To delete a row the method \texttt{delete(String table, String whereClause, String[] whereArgs)} is needed. It has the parameters that indicate the table and which row should be deleted.

\section{Multilingual Strings}

As described in section \ref{accessingresources} all non user edited names in the database are connected with the android \texttt{string.xml} resource. To achieve this connection a \texttt{HashMap} is needed which connects the resource names with the \texttt{id} \textit{Android} assigns to each string. It is important that the name in the database is exactly the same as in the string \texttt{xml}, otherwise the string won't be found and causes an error. It is also important to assure that every string key is unique to avoid false string being loaded. This map is called \texttt{resourcesIdMap} and has the name from the database as key and the associated resource id as value. Two methods that process this \texttt{HashMap}: \texttt{getStringResourceValueByResourceName} \texttt{(String resource Name)} and \texttt{getStringResourceNameByResourceValue (String resourceValue)}. The parameter \texttt{ResourceName} is the name that is described in the database and \texttt{ResourceValue} is the string from \texttt{strings.xml}.\\
Listing \ref{resourcebyName} shows the source code for reading the value of a resource with it's key. For example the key \texttt{'test\_string} returns as a result the string \texttt{'test'}.
\begin{lstlisting} [caption=\textit{getStringResourceValueByResourceName}, label=resourcebyName] 
public String getStringResourceValueByResourceName(String resourceName)
{
	int resID = context.getResources().getIdentifier(resourceName, "string", 
	context.getPackageName());
	String name = context.getResources().getString(resID);
	DataBaseHelper.resourcesIdMap.put(name, resID);
	return name;
}
\end{lstlisting}

\section{Load Project Icons}\label{impl:loadProjectIcons}

In the table \texttt{ProjectIcons} are the informations for all available icons stored. The important column for loading the icon is called \texttt{path}. It stores the relative path to the icon and the icon name. The method \texttt{loadProjectIcon} has to look if the icon exists at this position. As described in listing \ref{loadicon}, this icon is decoded in a \texttt{Bitmap} and returned to the caller. \texttt{Android} can present a \texttt{Bitmap} without further effort.

\begin{lstlisting} [caption=Method \textit{loadProjectIcon}, label=loadicon] 
public Bitmap loadProjectIcon(String path)
{
	AssetManager assetManager = this.context.getAssets();
	InputStream istr;
	Bitmap projectIcon = null;
	try
	{
		istr = assetManager.open(path);
		projectIcon = BitmapFactory.decodeStream(istr);
	} catch (IOException e)
	{
	}
	return projectIcon;
}
\end{lstlisting}
\section{Calculate Person Days}

This functionality is at the moment only adapted to the \textit{Function Point} technique. For calculating the \textit{person days} or \textit{man days} two methods are necessary. One is the \texttt{evaluateFunctionPointPersonDaysWithBaseProductivity}, when there are no terminated projects to calculate a \textit{points-per-day} value. \texttt{evaluateFunction- PointPersonDaysWithExistingProductivity} is called when there are already terminated projects which will help get the best fitting \textit{points-per-day} value. The calculation with the base productivity checks the database for the range of the total points of the project. If a project hast \textit{total points} of \textit{500}, for example, it fits in the base productivity that goes from \textit{350} to \textit{650} points, which has a \textit{points-per-day} value of \textit{16}. These \textit{500} points will then be divided with \textit{16} which results in \textit{31.25} \textit{man days}.\\
If there are enough terminated project the algorithm tries to calculate a more accurate \textit{points-per-day} value. The calculation checks all terminated projects for a project with less total points and for a project with more total points than the calculated project. Within these values the algorithm searches for those projects which total points are nearest to the calculated project. If one of these items are empty the algorithm selects the base productivity for this value instead. Listing \ref{evaldays} shows the code that is executed afterwards. For the smaller and bigger item the average \textit{points-per-day} are calculated. This value is then divided with the selected projects \textit{evaluated points} which returns the \textit{evaluated days} for the project.
\begin{lstlisting} [caption=Evaluate Days, label=evaldays] 
double averagePointsPerDay = (smallerItem.getPointsPerDay() + 
				biggerItem.getPointsPerDay()) / 2;
averagePointsPerDay = roundDoubleTwoDecimals(averagePointsPerDay);
evaluatedDays = roundDoubleTwoDecimals(project.getEvaluatedPoints() 
				/ averagePointsPerDay);
\end{lstlisting}
\section{Find related Projects}

The relation of two projects is calculated in the \texttt{ProjectRelationSolver} class. It is initialized with all projects that are listed in the database which inherits both active and deleted projects. Also the selected project for comparison is a parameter at initialization. The main method for finding related projects is \texttt{getRelatedProjects} and has the \texttt{relationBorder} as parameter which defines the percentage border for relation. It calls for each project the method \texttt{compareWithChosenProject} which compares a project with the selected one. As most of the property distances are save in the database, this method only needs to select the right distance from the database. Listing \ref{selectdistance} shows this selection with the development market distance as example. It calls the method \texttt{getPropertyDistance} from the \texttt{DatabaseHelper} and is a generalized method. It gets the table name of the property and the table name of the distance as parameter. Also the names of the columns for selecting the right value. The last two parameters are the values that have to be compared.
\begin{lstlisting} [caption=Development Market Distance, label=selectdistance] 
double developmentMarketDistance = databaseHelper.getPropertyDistance
			("DevelopmentMarkets", "DevelopmentMarketComparison", 
			"market_id_1", "market_id_2", "comparison_distance",
			project.getProjectProperties().getMarket(),
		    p.getProjectProperties().getMarket());

\end{lstlisting}
For \texttt{software architecture} and \texttt{programming language} these distance has to be calculated. This is done by loading the properties from the database and subtracting them. These distances are then added to the \texttt{distanceSum} which is then calculated with the equation in listing \ref{calcdistance}. For logging purposes the calculation was separated into two parts. Before returning the \texttt{differencePercentage} has to be subtracted from \textit{100} to get the relation.
\begin{lstlisting} [caption=Percentage Difference Calculation, label=calcdistance] 
differencePercentage = databaseHelper.roundDoubleTwoDecimals(distanceSum 
										/ 103) * 100;
\end{lstlisting}
\section{ListView Elements and ViewHolder}

The difficulty with \texttt{ListViews} was to achieve that each item remember his content and can be processed. For example selecting a \textit{button} in the list element which returns the name of the item. This was achieved by using so-called \texttt{ViewHolder}. These are just sub-classes that inherit all \texttt{View} elements of the list item. This is done in the adapter of \texttt{ListViews}. The method \texttt{getView} is called for filling the \texttt{ListView} with items. One parameter of this method is the \texttt{convertView} which represents the actual item. The trick to achieve that each element remembers its content to add the \texttt{ViewHolder} as a tag to the \texttt{convertView}. Listing \ref{lvholder} shows an example of this allocation. For each list item a new \texttt{ViewHolder} is initialized. Each including \texttt{View} element has to be assigned with the element inside the \texttt{convertView}. Finally the \texttt{holder} is assigned as a tag to the \texttt{convertView}.
\begin{lstlisting} [caption=ListView Holder example, label=lvholder] 
convertView = vi.inflate(R.layout.
	function_point_influence_factorset_with_subitems_list_item, null);
holder = new ViewHolder();
holder.tvShowSubitems = (TextView) convertView.findViewById(
	R.id.tvShowSubitems);
convertView.setTag(holder);
\end{lstlisting}
