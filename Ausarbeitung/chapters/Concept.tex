\chapter{Application concept}

In diesem Kapitel sollen die Konzepte die in der Anwendung umgesetzt wurden besprochen werden


\section{Project planning}

Herangehensweise an die Planung, Regelmäßige Meetings mit Betreuer, Allgemeine Features von vorhandener Software Analysiert, neue Features Entwickelt, Im Gespräch mit Rock Featurewünsche besprochen, als nächstes Ziele und Anforderungen bestimmt

\subsection{Objectives}

Auflistung der wichtigsten Ziele, Besprechen der Ziele und wofür diese wichtig sind

\subsection{Requirements}

Herangehensweise. 
Die Schwierigsten Anforderungen heraussuchen und besprechen

\subsection{Cost Estimation}

Schätzung der eigenen Anforderungen mit Function Point

\section{Architecture}

Schaubild zur Architektur der Anwendung, einzelne Punkte besprechen

\section{Components}

Beschreibung in Welche Komponenten die Anwendung aufgeteilt ist.

\subsection{Database Helper}

Alle SQL Befehle werden über diese Klasse gemacht. Muss in allen Klassen die Zugriff auf die Datenbank haben wollen eingebunden werden. Befehle zu großen Teilen sehr Abstrakt gehalten

\subsection{Influence Factors}

Kompontene mit den Einflussfaktoren entsprechend der gewählten Schätzmethode. Summe Aller Faktoren zur Berechnung mit den Punkten der Schätzmethode

\subsection{Projekt Daten}

Speicherung aller benötigten Daten in einer Klasse. Properties und Einflussfaktor in eugener Klasse und als Objekt in Projekt vorhanden

\subsection{Related Project}

Berechung der Relevanz verschiedener Projekte. Eingehen auf die Grundlage hierfür und Beschreibung der Berechnung 

\subsection{Weitere Features}

Export, statistic, Help DB, Feedback, Project Filter, Analysis


\section{Database design}

Vorheriges Design der zentralen Datenbank. Wichtig um die Klassen anzupassen und vorher schon mit wichtigen Daten zu füllen

\subsection{Project Database}

Datenbank für alle Projekte, Eigenschaften, Einflussfaktoren

\subsubsection{Project Properties}

Welche Tabellen gibt es. Wichtige Tabellen, Aufbau der Tabellen und Grund

\subsubsection{Influence Factors}

Wie die Einflussfaktoren Aufgebaut, was ist der Gedanke dazu?

\subsubsection{Projects}

Wie sind Projekte gespeichert, Aufteilung in Projekt, Projektdetails und zugehörige Tabellen, wie die Schätzung Organisiert und wie der Zugriff auf die Elemente

\subsection{Userinformation Database}

Datenbank für Spätere Synchronisation und Userinformationen vom Server, Konzept dazu

\section{User Interface}

\subsection{Projects Overview}

Anordnung der Projekte, Wichtige Informationen zum sehen, Filtern und Suchen nach Projekten

\subsection{Project Creation}

Komponente zur korrekten Erstellung von Projekten, Geführte Eingabe, Korrekte Erstellung von Projekten in der DB, Swipe Funktion

\subsection{Estimate Function Point Project}

Aufbau der Schätzung, Umwandlung von Tabelle in App

\subsection{Influence Factors}

Aufbau der EInflussfaktoren, Neue anlengen

\subsection{Analysis}

Wie die Analyse aufgebau und was soll diese bringen?

\section{Adjusted Estimation Process}



