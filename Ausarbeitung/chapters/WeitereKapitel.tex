\chapter{Weitere Kapitel}

Die Gliederung h�ngt nat�rlich vom Thema und von der L�sungsstrategie ab. Als n�tzliche
Anhaltspunkte k�nnen die Entwicklungsstufen oder - schritte z.B. der Softwareentwicklung betrachtet werden. N�tzliche Gesichtspunkte erh�lt und erkennt man, wenn man sich
\begin{itemize}
  \item in die Rolle des Lesers oder
  \item in die Rolle des Entwicklers, der die Arbeit z.B. fortsetzen, erg�nzen oder pflegen soll,
\end{itemize}
versetzt. In der Regel wird vorausgesetzt, dass die Leser einen fachlichen Hintergrund haben - z.B. Informatik studiert haben. D.h. nur in besonderen, abgesprochenen F�llen schreibt man in popul�rer Sprache, so dass auch Nicht-Fachleute die Ausarbeitung prinzipiell lesen und verstehen k�nnen.

Die �u�ere Gestaltung der Ausarbeitung hinsichtlich Abschnittformate, Abbildungen, mathematische Formeln usw. wird in \hyperref[Stile]{Kapitel~\ref*{Stile}} kurz dargestellt.