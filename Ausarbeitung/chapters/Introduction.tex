\chapter{Introduction}

%Thema
Most of the contracts IT companies subscribe are projects and these are notorious for going past their deadline and over their budget. According to the study of Capgemini in 2014 \cite{capgemini}, the importance of cost estimation increases every year. The study asked for the most important requirements in the IT for the next years. Top requirement, as asked in the study, is to increase the efficiency, which means to lower the costs and to meet determined deadlines. This will increase the effort companies have to take in planning their projects.\\
%Fokus / Aspekt
All businesses want to lower the risk of delayed or canceled projects. This means IT companies have to take more effort in requirements engineering and cost estimation to give their clients an accurate estimation of the upcoming project. This results in a increasing effort for requirements engineering in IT projects. As cost estimation is a part of requirements engineering, it will lower the chance of a budget overrun. To achieve this 6\% to 12\% of the project time has to be spend in requirements engineering. The budget overrun is limited with this spent time to a maximum of 50\% of the estimated cost\cite{Partsch}. Which means, that a higher effort in requirements engineering and cost estimation will lower the chances of failed projects.\\
Therefore cost estimation is an important element for planning software projects and can be responsible for successful or failed projects. It is even more important to estimate as precisely as possible to guide the project to success. There are several methods for these estimations that can be used at different phases of the project. These methods of estimation lean their result on the information they get from the development process and the artifacts of the particular project phase. These include requirement documents, diagrams or the program code itself. All available artifacts depend on the used process model and the project phase \cite{EntwicklungKompakt}. Based on the described information, the actual project can be categorized, so that the \textit{'best fitting'} estimation method for the current estimation. These methods can be time-consuming and related projects can often only be found in the own company context or are based on experiences.\\
%Methode
This paper aims to develop a mobile application which supports the Function Point estimation. The application aims to make the estimation process in IT-projects simpler and more efficient. To achieve a better way for estimating costs the most important design guideline was \textit{'Only show what I need when I need it'} \cite{materialdesign}. The comparison between projects has to be formalized and implemented. The idea is to give the user an overview over terminated projects and how they were estimated. The user has the possibility to transfer such an estimation to a new project or get a quick view how much man days it took.\\
%Ergebnisse
The developed application \textit{MobileEstimate} should prove that cost estimation on mobile devices is possible. It should allow to estimate the costs of a project with function point method and among the existing projects related projects should be displayed. Estimation results from a related project are possible to be viewed in the application and transfered to another estimation.\\
%Schlussfolgerung - Ins Fazit
%From an computer scientist viewpoint, the conclusion to be drawn with the implemented application with the Android Design Principles is possible and allows a simpler way to estimate the costs of IT projects. To make the application marketable, plenty still remains to be done and the use of the application in "Spezifikation interaktiver Systeme", at the University of Applied Science in Trier, will give more feedback about the application and what additional features are needed.\\
