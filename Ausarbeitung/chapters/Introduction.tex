\chapter{Introduction}

%Thema
Most of the contracts IT companies subscribe are projects and these are notorious for going past their deadline and over their budget. According to the study of Capgemini in 2014 \cite{capgemini}, the importance of cost estimation increases every year. The study asked for the most important requirements in the IT for the next years, with the top requirement to increase the efficiency, which means to lower the costs and to meet determined deadlines. This will enhance the effort companies have to take in planning their projects.\\
%Fokus / Aspekt
All businesses want to lower the risk of delayed or canceled projects. This results in more effort IT companies have to take in requirements engineering and cost estimation to give their clients an accurate estimation of the upcoming project. As a result of the increase in requirements engineering and, subsequently, the cost estimation, to 6\% till 12\% the project time will lower to cost overrun to a maximum of 50\% \cite{Partsch}.\\
Therefore cost estimation is an important element for planning software projects and can be responsible for successful or failed projects. It is even more important to estimate as precisely as possible to guide the project to success. There are several methods for these estimations that can be used at different phases of the project. These methods of estimation lean their result on the information they get from the development process and the artifacts of the particular project phase. These include requirement documents, diagrams or the program code itself. All available artifacts are depending on the used process model and the project phase \cite{EntwicklungKompakt}. Based on the described information a categorization for the actual project can be made, to find the “best fitting” estimation method for the current estimation. These methods can be time-consuming and related projects can most times only be found in the own company context or are based on experiences.\\
%Methode
Objectives of the proposal is to develop a mobile application which support the function point estimation. The implemented process aims to makes the estimation process in IT-projects simpler and more efficient. To achieve a better way for estimating costs the most important design guideline was "Only show what I need when I need it" \cite{materialdesign}. The comparison between projects has to be formalized and implemented. This should give the user an overview over terminated projects and how they were estimated. Either to transfer the estimation to the new project or to get a quick view how much days it took.\\
%Ergebnisse
The application MobileEstimate proves that cost estimation on mobile devices is possible. Its possible to estimate the costs of a project with function point method and among the existing projects related projects can be displayed. Estimation results from a related project can be viewed in the application and transfered to another estimation. Also a complete estimation can be exported as an excel file and processed afterwards.\\
%Schlussfolgerung
From an computer scientist viewpoint, the conclusion to be drawn with the implemented application with the Android Design Principles is possible and allows a simpler way to estimate the costs of IT projects. To make the application marketable, plenty still remains to be done and the use of the application in the module "Spezifikation interaktiver Systeme" will give more feedback about the application and what additional features are needed.\\
