%%%%%%%%%%%%%%%%%%% vorlage.tex %%%%%%%%%%%%%%%%%%%%%%%%%%%%%
%
% LaTeX-Vorlage zur Erstellung von Projekt-Dokumentationen
% im Fachbereich Informatik der Hochschule Trier
%
% Basis: Vorlage svmono des Springer Verlags
%
%%%%%%%%%%%%%%%%%%%%%%%%%%%%%%%%%%%%%%%%%%%%%%%%%%%%%%%%%%%%%

\documentclass[envcountsame,envcountchap, deutsch]{i-studis}

\usepackage{makeidx}         	% Index
\usepackage{multicol}        	% Zweispaltiger Index
%\usepackage[bottom]{footmisc}	% Erzeugung von Fußnoten

%%-----------------------------------------------------
%\newif\ifpdf
%\ifx\pdfoutput\undefined
%\pdffalse
%\else
%\pdfoutput=1
%\pdftrue
%\fi
%%--------------------------------------------------------
%\ifpdf
\usepackage[pdftex]{graphicx}
\usepackage[pdftex,plainpages=false]{hyperref}
%\else
%\usepackage{graphicx}
%\usepackage[plainpages=false]{hyperref}
%\fi

%%-----------------------------------------------------
\usepackage{color}				% Farbverwaltung
\usepackage{ngerman} 			% Neue deutsche Rechtsschreibung
\usepackage[english]{babel}
%\usepackage[latin1]{inputenc} 	% Ermöglicht Umlaute-Darstellung
\usepackage[utf8]{inputenc}  	% Ermöglicht Umlaute-Darstellung unter Linux (je nach verwendetem Format)

%-----------------------------------------------------
\usepackage{listings} 			% Code-Darstellung
\lstset
{ 
	basicstyle=\scriptsize, 	% print whole listing small
	keywordstyle=\color{blue}\bfseries,
	% underlined bold black keywords
	identifierstyle=, 			% nothing happens
	commentstyle=\color{red}, 	% white comments
	stringstyle=\ttfamily, 		% typewriter type for strings
	showstringspaces=false, 	% no special string spaces
	framexleftmargin=7mm, 
	tabsize=3,
	showtabs=false,
	frame=single, 
	rulesepcolor=\color{blue},
	numbers=left,
	%linewidth=146mm
	xleftmargin=8mm,
	captionpos=b
}

\usepackage{textcomp} 			% Celsius-Darstellung
\usepackage{amssymb,amsfonts,amstext,amsmath}	% Mathematische Symbole
\usepackage[german, ruled, vlined]{algorithm2e}
\usepackage[a4paper]{geometry} % Andere Formatierung
\usepackage{bibgerm}
\usepackage{array}
\usepackage{chngcntr} 
\numberwithin{table}{subsection} %add tables to nummeration
%\numberwithin{equation}{subsection}
\makeatletter
\AtBeginDocument{%
	\let\c@figure\c@lstlisting
	\let\thefigure\thelstlisting
	\let\ftype@lstlisting\ftype@figure % give the floats the same precedence
}
\makeatother

\hyphenation{Ele-men-tar-ob-jek-te  ab-ge-tas-tet Aus-wer-tung House-holder-Matrix Le-ast-Squa-res-Al-go-ri-th-men vor-ge-schla-gen Pro-jekt-kos-ten   evaluate-Function-Point-Person-Days-With-Existing-Productivity} 		% Weitere Silbentrennung bei Bedarf angeben
\setlength{\textheight}{1.1\textheight}
\pagestyle{myheadings} 			% Erzeugt selbstdefinierte Kopfzeile
\makeindex 						% Index-Erstellung


%--------------------------------------------------------------------------
\begin{document}
%------------------------- Titelblatt -------------------------------------
\title{Mobile Anwendung für die Kostenschätzung mit Android}
\subtitle{Mobile Application for Cost Estimations in Android}
%---- Die Art der Dokumentation kann hier ausgewählt werden---------------
%\project{Bachelor-Projektarbeit}
\project{Bachelor-Abschlussarbeit}
%\project{Master-Projektstudium}
%\project{Master-Abschlussarbeit}
%\project{Seminar zur Vorlesung ...}
%\project{Hausarbeit zur Vorlesung ...}
%--------------------------------------------------------------------------
\supervisor{Prof. Dr. Georg Rock} 		% Betreuer der Arbeit
\author{Oliver Fries} 							% Autor der Arbeit
\address{Trier,} 							% Im Zusammenhang mit dem Datum wird hinter dem Ort ein Komma angegeben
\submitdate{29.02.2016} 				% Abgabedatum
%\begingroup
%  \renewcommand{\thepage}{title}
%  \mytitlepage
%  \newpage
%\endgroup
\begingroup
  \renewcommand{\thepage}{Titel}
  \mytitlepage
  \newpage
\endgroup
%--------------------------------------------------------------------------
\frontmatter 
%--------------------------------------------------------------------------
%\preface

Ein Vorwort ist nicht unbedingt 				% Vorwort (optional)
%\kurzfassung

%\newcommand{\kurzfassung}[1][Abstract]{\chapter*{#1}\markboth{#1}{#1}}
%\kurzfassung
%\newpage

\chapter*{Abstract}

%\begin{abstract}
%% deutsch
\paragraph*{}
Lorem Ipsum

%% englisch
\paragraph*{}
Lorem Ipsum

%\end{abstract} 			% Kurzfassung Deutsch/English
\setcounter{tocdepth}{1}
\tableofcontents						% Inhaltsverzeichnis
%\listoffigures 							% Abbildungsverzeichnis (optional)
%\listoftables 							% Tabellenverzeichnis (optional)
%--------------------------------------------------------------------------
\mainmatter                        		% Hauptteil (ab hier arab. Seitenzahlen)
%--------------------------------------------------------------------------
% Die Kapitel werden in separaten .tex-Dateien abgelegt und hier eingebunden.
\chapter{Introduction}

%Thema
Most of the contracts IT companies subscribe are projects and these are notorious for going past their deadline and over their budget. According to the study of Capgemini in 2014 \cite{capgemini}, the importance of cost estimation increases every year. The study asked for the most important requirements in the IT for the next years. Top requirement, as asked in the study, is to increase the efficiency, which means to lower the costs and to meet determined deadlines. This will increase the effort companies have to take in planning their projects.\\
%Fokus / Aspekt
All businesses want to lower the risk of delayed or canceled projects. This means IT companies have to take more effort in requirements engineering and cost estimation to give their clients an accurate estimation of the upcoming project. This results in a increasing effort for requirements engineering in IT projects. As cost estimation is a part of requirements engineering, it will lower the chance of a budget overrun. To achieve this 6\% to 12\% of the project time has to be spend in requirements engineering. The budget overrun is limited with this spent time to a maximum of 50\% of the estimated cost\cite{Partsch}. Which means, that a higher effort in requirements engineering and cost estimation will lower the chances of failed projects.\\
Therefore cost estimation is an important element for planning software projects and can be responsible for successful or failed projects. It is even more important to estimate as precisely as possible to guide the project to success. There are several methods for these estimations that can be used at different phases of the project. These methods of estimation lean their result on the information they get from the development process and the artifacts of the particular project phase. These include requirement documents, diagrams or the program code itself. All available artifacts depend on the used process model and the project phase \cite{EntwicklungKompakt}. Based on the described information, the actual project can be categorized, so that the \textit{'best fitting'} estimation method for the current estimation. These methods can be time-consuming and related projects can often only be found in the own company context or are based on experiences.\\
%Methode
This paper aims to develop a mobile application which supports the Function Point estimation. The application aims to make the estimation process in IT-projects simpler and more efficient. To achieve a better way for estimating costs the most important design guideline was \textit{'Only show what I need when I need it'} \cite{materialdesign}. The comparison between projects has to be formalized and implemented. The idea is to give the user an overview over terminated projects and how they were estimated. The user has the possibility to transfer such an estimation to a new project or get a quick view how much man days it took.\\
%Ergebnisse
The developed application \textit{MobileEstimate} should prove that cost estimation on mobile devices is possible. It should allow to estimate the costs of a project with function point method and among the existing projects related projects should be displayed. Estimation results from a related project are possible to be viewed in the application and transfered to another estimation.\\
%Schlussfolgerung - Ins Fazit
%From an computer scientist viewpoint, the conclusion to be drawn with the implemented application with the Android Design Principles is possible and allows a simpler way to estimate the costs of IT projects. To make the application marketable, plenty still remains to be done and the use of the application in "Spezifikation interaktiver Systeme", at the University of Applied Science in Trier, will give more feedback about the application and what additional features are needed.\\

\chapter{Theoretical Background}

This chapter describes the fundamentals for the understanding of this paper. The cost estimation process in IT projects and the different methods to calculate the cost of a project are described in this chapter. The state of the art report combined with the market description will give a short overview over the situation about software estimation tools on the market and the possibility of a mobile solution of cost estimations. Android as the chosen platform and Java as the programming language will not be described in detail here and are assumed to be known.

\section{Cost estimation in software engineering}

The most expensive components of computer systems are software products. While private clients are mostly interested in the final price of a product, business clients of IT companies typically want to know the costs of the software before project launch. As analyzed in the \textit{IT-Trends} study from Capgemini \cite{capgemini} the IT budget of companies is growing up to 10\% every year. Whether developing a new project or standardizing existing software, the project costs are always on the main focus by the project management. As human resources is the biggest part of software costs, project managers and especially business clients want to know the estimated spendings and completion time of a project. Most of the estimation methods focus on this aspect and give the result in man days. These estimated days can then be converted into the real costs.
\\
Basically the cost estimation in software engineering wants to answer following questions:
\begin{enumerate}
\item How much effort is required to complete the project?
\item How much days are needed to complete the project?
\item What is the total cost of the project?
\end{enumerate}
While projects are a living thing, the effort may change due to unexpected difficulties. For a precise estimation of the total costs an adjustment cannot be avoided and it can be useful to change the estimation method in a later project phase. This means that the estimation process is not an one-time thing but will change through the life-time of a project. 


\subsection{Estimation Process}

It is common to create the first cost estimation before the system design, but also for monitoring purposes, milestones or if the client wants an overview of the project. Each time an actual cost estimation is needed the estimation process is executed which is a set of techniques and procedures that are used to derive the software cost estimate. Kathleen Peters described the basic process of an estimation, as seen in figure \ref{fig:basicEstimationProcess}, as it is common in the industry \cite{estimationProcess}.
As can bee seen from figure \ref{fig:basicEstimationProcess}, there are seven steps in the estimation process. The first part is to collect the initial requirements which is essential to know what the project is about and evaluate the approximate project size. With an selected estimation method, which are described in section \ref{chapter:estimationmethods}, the evaluated size of the project is then estimated. Afterwards the effort in man-days is calculated from which the cost schedule is created. Data from older projects can be included into the cost estimation. In the process step to approve the estimation, it has to be decided, if the costs are acceptable or if range of functions has to be shortened and the re-estimation has to bee started. If the cost estimation is acceptable the development of the product can start or continue.\\
\begin{figure}[h] 
	\centering 
	\includegraphics[width=13cm]{images/estimationProcess.PNG} 
	\caption{- The Basic Project Estimation Process}
	Source: Peters, Kathleen - Software Project Estimation, Page 3  
	\label{fig:basicEstimationProcess}
\end{figure}\\
In this classical view of the estimation process there are four outputs generated which can be described as following:
\begin{enumerate}
	\item Actual Size - the size of the project as a numerical value to make it comparable.
	\item Manpower Loading - the amount of personnel that is allocated to the project.
	\item Project Duration - the time that is needed to complete the project.
	\item Effort - the amount of effort required to complete the project is usually measured in units as man-days (MD) or person-months (PM).
\end{enumerate}
As described before, the estimation process can be triggered at any time in the project to re-estimate the costs. Depending on the project stage another estimation method than used before can be more precise.\\
The overview shown in fig. \ref{fig:estimationMethodInStage} shows that for example the SLIM method, is more suitable at the beginning of a project, whereas the ZKP method is more suitable after the system design stage. Most of the estimation methods can be used after the study stage. This is because a rough overview of the project size exists after the study stage. Different estimation methods may also change the evaluation output, which is one of the difficulties of cost estimations.\\
\begin{figure}[h] 
	\centering 
	\includegraphics[width=13cm]{images/Einsatzzeitpunkte2.PNG} 
	\caption{- Starting Points of Estimation Methods} 
	Source: \url{http://winfwiki.wi-fom.de/index.php/Methoden_und_Verfahren_der_Aufwandsch\%C3\%A4tzung_im_Vergleich}
	\label{fig:estimationMethodInStage}
\end{figure}

\subsection{Difficulty of estimations}

One of the problems in estimating costs is that the actual code is only a small part of the project. Beside project planning there are many administrative tasks to do, like coordination of the project or searching and fixing errors. Most estimation methods evaluate the estimated time for the implementation only and too little or no time is evaluated for the non implementation tasks. This resolves in an underestimation of the non implementation tasks or an overestimation if they are predicted as a high value \cite{itplanung}.
\\
Most project managers rely on their experience from past projects. This is an advantage, but as technology changes fast and new projects inherit new problems there is no prior experience for some parts of the project. That can make experiences from prior projects useless. Because of the unique nature of projects it is common that a new project has big parts where no experience exists. Another difficulty of estimations is the fact that people who are inherited in the project have a more positive outlook and mostly underestimate the costs.
Also, customers often got a target time for the project, which leads in adjustment of the cost estimation. This leads to budget overrun if the needed resources are not available for the project \cite{winfwiki}.
\\
It is also not guaranteed that the estimated costs are accurate and stay within the budget. An estimation can easily go past their estimated target as new technologies and unexpected difficulties are commonplace. A partial requirements engineering can also cause an inaccurate cost estimation due to unexpected difficulties in the implementation.\\

\subsection{Methods for estimation}\label{chapter:estimationmethods}

Estimation methods are different metrics to calculate the costs of a project and there are different approaches to categorize the estimation methods. The categorization used in this paper is based on the book "Management von IT-Projekten" by Hans W. Wieczorrek, where the methods are subdivided in algorithmic, comparative, operating figures and expert discussion \cite{itplanung}.\\
All estimation techniques have in common, that only with a combined use of these different estimation methods a suitable result can be achieved as a measured value for the project. For the evaluation of the needed effort the underlying metric is used to calculate the effort for the project. On the basis of charge rates the effort size is calculated out of it.\\

\subsubsection{Algorithmic Method}

The algorithmic method uses a closed formula, which is based on empiric evaluation of already terminated projects or on existing mathematically models. Different forms of this method are the weight and the sampling method, which only differ in their usage.\\
The accuracy of the estimation depends primarily on the precision of the influence factors \cite{itplanung}. The algorithmic method always connects measurable project sizes, such as lines of code and implementable features, with influencing factors to get the result, represented as required effort in personnel costs. The basic formula, as described by Wieczorrek \cite{itplanung}:

\begin{equation}
	Personnel costs = f(result quantity, influencing factors)
\end{equation}

\subsubsection{Comparative Method}

Not based on a formula or numerical connection, the comparative method tries to create a reference between the current project and past projects. Therefore projects from the own company or the same industry sector are analyzed with appropriate comparison methods. This estimation method has the advantage, that it can be used early in project development \cite{itplanung}. This method can be used for hardware and software projects.

\subsubsection{Key Figures Method}

Estimation methods based on key figures can be differed to multiplier and percentage  method. The multiplier method uses units of power as the base to estimate the total expenditure, whereas the percentage method uses the effort of a project stage to estimate the effort for the next stage.\\
After project completion a post calculation determines the total project costs and the amount of specific types of costs. In order to calculate these costs, they will be divided by the scope of the developed product. This results in new key figures which can be used for new projects by multiplication of the estimated scope with the appropriate key figures. Regular actualization of the key figures is necessary for right results \cite{itplanung}.\\

\subsubsection{Expert Discussion Method}

As a quantitative and heuristic method the expert discussion uses knowledge from selected groups of people. It differentiates between four kinds: single person interview, multiple person interview, Delphi method and assessment meeting.\\
The advantage of this estimation method is, that they are useful for all project types. But they have the risk of a strongly affection by subjective opinions and the experience of the interviewees. As a result, expert discussions should never be used for complete projects but only for sub projects \cite{itplanung}.\\

\section{Estimation Techniques}

The existing estimation techniques rely on experience-based judgments by project managers
who created, by combining estimation methods, techniques that can be used to estimate the effort of a project. Most of the estimation techniques became popular in the 80's. With the agile projects becoming more popular estimation techniques for these project types are rising. \\
Because of the fundamental uniqueness nature of projects an 'universal, everywhere applicable and always delivering the correct estimation' technique does not exist, according to Litke \cite{litke}. There is also no clear selection progress for the estimation technique to use, beside the time aspect when the use of a technique is available. As shown in figure \ref{fig:estimationMethodInStage}, the function point and the COCOMO technique are both possible after the study stage. As a comparative method, the function point technique has not much in common with COCOMO, which is an algorithmic technique. The project manager has to balance the weigh of each technique and probably choose that one he has most experience with. As an example for estimation techniques these will be described here in more detail with a comparison at the end.


\subsection{Function Point} \label{FPMethod}

The Function Point technique was first mentioned by Allan J. Albrecht in 1979 at the IBM symposium \cite{albrecht}. He declared that an useful measurement of productivity is only possible in relation to the functionality that is visible to the user. This measured productivity needs to be independent of the used technology and is calculated with the proportion of project effort and the allocated function points.\\
This resulted in the idea to turn over this calculation for a preliminary estimation of the effort the project would have. Because of the clarity and the flexibility the technique spread fast. \\
It helps to estimate the scope of a project to an early stage and is suitable for benchmarking in the own company as well as on national or international level \cite{FPKompakt}. It contains algorithmic and also comparative methods. Basically, this technique uses five steps for estimation \cite{jenny}:
\begin{enumerate}
	\item Determining the components
	\item Evaluation of the components
	\item Calculating the function points
	\item Categorization of the influence factors
	\item Calculating the development effort with the function points and influence factors
\end{enumerate}
The most important part of this technique is that all measurements only include the user view. This means that the users view is focused on that functions that are important for the specific business process. Implemented business processes are the components that have to be determined.

\subsubsection{Determining the Components}\label{fpcomponents}

To divide the project in useful components, all inherited business processes are divided into elementary process. These are the smallest and from business perspective view useful and closed activities, that can be performed by the system \cite{FPKompakt}. It is useful to categorize the components, because a change in the datasets is followed by more effort than changing a request \cite{itplanung}. This distinction is made into five categories:\\
\begin{enumerate}
	\item Input Data
	\item Output Data
	\item Request
	\item Dataset
	\item Reference Data
\end{enumerate}
The \textit{Input Data}, sometimes called \textit{External Inputs} (EI), is an elementary process in which data crosses the boundary from the outside of the application to the inside. This data comes from a data input screen or another application. To maintain one or more logical files, this data can be used for or to control business informations.\\
\textit{External Outputs} (EO), or Output Data, are an elementary process in which derived data passes across the boundary from the inside to the outside. Created output files or reports are sent to other applications. These are created from one or more internal logical files and external interface files.\\
Internal \textit{Logical Files} (ILF’s) or Dataset are an user identifiable group of logically related data that resides entirely within the applications boundary and is maintained through external inputs.\\
The next category are requests or \textit{External Inquiry} (EQ) or Request. An elementary process with both input and output components that result in data retrieval from one or more internal logical files and external interface files. This process does not update any Internal Logical files, and the output side does not contain derived data.\\
The last category is the \textit{Reference Data} or \textit{External Interface Files} (EIF), which are a user identifiable group of logically related data that is used for reference purposes only. This data resides entirely outside of the application and is maintained by another application \cite{fpafundamentals}.\\
The result of the categorization is an amount of components for each category. This categories must then be evaluated.

\subsubsection{Evaluation of the Components}

The evaluation of the components is simply the classification of each category in their level of difficulty (simple, medium or complex). After this, each component is multiplied with the point value according to their difficulty. \\
The following table is described by Wieczorrek and are standard values for the function point technique \cite{fpafundamentals}:\\
\begin{table}[h] 
	\centering 
	\setlength{\tabcolsep}{4pt}
	\begin{tabular}{|l|c|c|c|}\hline
		Category & simple & medium & complex \\ \hline
		Input Data & 3 & 4 & 6\\ \hline
		Output Data & 4 & 5 & 7\\ \hline
		Request & 3 & 4 & 6\\ \hline
		Dataset & 7 & 10 & 15\\ \hline
		Reference Data & 5 & 7 & 10\\ \hline
	\end{tabular}
	\caption{Point value of each category} 
	\label{tab:pointvalues} 
\end{table} \\

\subsubsection{ Calculating the amount of Function-Points}

After the evaluation there is an amount of components for each of the categories from above. To get the number of function points each category has to be multiplied according to the selected weigh and all categories are summed. This results in the following equation:

\begin{equation}
	\textit{E1} =  \sum \limits_{1}^n  (\textit{Function} \cdot \textit{Difficulty}) \label{fp:E1}
\end{equation}

\textit{Function} is the respective category and \textit{Difficulty} is the weigh from table \ref{tab:pointvalues}. The resulted value E1 is necessary for evaluating the estimated points with the influence factor.

\subsubsection{ Classification of Influence Factors}\label{fp:classificationInfluence}

Do get a more realistic estimation, all influences that can affect the project surroundings are measured. There are seven defined influence factors \cite{Softwaremanagement}, which are described below.\\
Each of the influence factor has a value between zero and five which describes how much the factor influences the project.\\

\begin{enumerate}
	\item \textbf{Integration into other applications}\\The system will work with different applications and will send and receive data from other applications. This states if there is a cooperation with other applications and if the communication exists online or offline.
	\item \textbf{Local Data Processing}\\This factor describes if the system will work with distributed data. Zero means that the system does not work with other applications and five means that there is an integration into other applications in both ways.
	\item \textbf{Transaction Rate}\\A high transaction rate affects planing, development, installation and maintenance of the system. It describes how much transactions are to expected with the system.
	\item \textbf{Processing Logic}\\The processing logic can be divided in 4 subcategories: \textit{Arithmetic Operation}, \textit{Control Procedure}, \textit{Exception Regulation} and \textit{Logic}. Arithmetic Operation describes the intensity of the operations in the project. The controlling of the results is stated within the Control Procedure. The \textit{Exception Regulation} describes how eventual exceptions are treated and the Logic multiplier describes how much effort is to be expected for planning the logical component of the project.
	\item \textbf{Reusability}\\ How much of the produced software has to be reusable in other projects. A high value means extra effort in the planning stage for module-based development.
	\item \textbf{Stock Conversion}\\ This describes how much of the used data needs to be transformed for the usage within the project. A high value means, that much input data from other applications have to be transformed into data the application can process.
	\item \textbf{Facilitate Change}\\ The application was especially planned and developed in such a way that changes can be made easily. A high value means that the user can make changes on the system on its own and that the changes are available immediately.
\end{enumerate}
When all influence factors are set, all values for each factor will be summed up to the value E2. 

\begin{equation}
\textit{E2} =  \sum \limits_{1}^n   \textit{Influence Factor}  \label{fp:E2}
\end{equation}\\
This influence factor indicator has then to be transformed to a multiplier that calculable with function points \cite{Softwaremanagement}\cite{fpafundamentals}.  

\begin{equation}
	\textit{E3} =\frac{\textit{E2}}{100}  + 0,7 \label{fp:E3}
\end{equation}

\subsubsection{Calculation of the Function Points}

With the calculated Function Points (E1) and the Influence Factor multiplier (E3) are now the total-function-points (TFP) can now be calculated with the following formula:
%equation,formula
\begin{equation}
	\textit{TFP} = \textit{E1} \cdot \textit{E3}  \label{fp:TFP}
\end{equation}\\
From the regression analysis of previous projects a standard calculation of Function Points per day can be made using table \ref{tab:pointsperday}.\\
The project has to be classified with their Total-Function-Points and divided through the appropriate points per day, as described in \ref{tab:pointsperday}. This results in the expected man days for this project.\\
\begin{table}[h] 
	\centering 
	\setlength{\tabcolsep}{4pt}
	\begin{tabular}{|l|c|c|}\hline
		Estimated Size    & Function Points & Points per Day\\ \hline
		Small Project     & till 350        & 18 \\ \hline
		Mid Small Project & till 650        & 16 \\ \hline
		Medium Project    & till 1100 		& 14 \\ \hline
		Mid Large Project & till 2000 		& 12\\ \hline
		Large Project     & as of 2000 		& 10 \\ \hline
	\end{tabular}
	\caption{Function Points to Days} 
	\label{tab:pointsperday} 
\end{table} 

\subsection{COCOMO} \label{COCOMOMethod}

The Constructive Cost Model (COCOMO) is used when it is not possible to rely on experience. It is based on algorithmic and parametric methods that merge parameters from software projects, combining the system size, product properties, project and team factors with the effort for developing the system \cite{jenny}.\\
As an public domain model it is free to use and is considered to be a classic for algorithmic methods. This technique was adjusted to the IT development through the years and is common in many companies for cost estimation.
The accuracy of the COCOMO techniques rises with later project stages. The deviation of the cost estimation is at the beginning of the project between \textit{0,25 * MD} and \textit{4 * MD}. In later project stages this variation decrease until it is zero just before the project ending \cite{sommerville}.\\
The parameters for COCOMO can be split into three parts. The project classes, model variants and the implementation time and effort.

\subsubsection{Project Classes}

The project classes represent the estimated size of the program itself. These are expressed in kilo delivered source instructions (KDSI). COCOMO classifies three project classes with different calculation factors as described in table \ref{tab:projectclasses} \cite{sommerville}.

\begin{table}[h]
	\centering 
	\setlength{\tabcolsep}{4pt}
	\begin{tabular}{|l|c|c|p{6cm}|}\hline
		Complexity	& Calculation Factor& KDSI 	& Describtion\\ \hline
		Small   	& 1.05        		& $<$ 50  			& A small, well-known project team works together, the environment is well-known, there is no big innovation necessary and no pressure due to a deadline.\\ \hline
		Medium 		& 1.12        		& 50 - 300 			& Employees with average experience, team members with some experiences in subjections are working together.  \\ \hline
		Complex 	& 1.20 				& $>$ 300 			& High cost and deadline pressure, high innovations and an extensive project. High requirements to the project team and new components.  \\ \hline
	\end{tabular} 
	\caption{COCOMO project classes} 
	\label{tab:projectclasses} 
\end{table}

\subsubsection{Model Variants}

The COCOMO technique can be differed into basis, intermediate and detailed model. These represent the detail level of the cost estimation.\\
The first stage is also called basis estimation and is a rough estimation, which estimates the project costs to an early stage of the project. The costs are calculated with an equation without dividing the project into structure or time aspects. The base model is an useful starting point for later estimations.\\
The second stage adjusts the first estimation to a higher level of detail by differentiating the development stages. It is not a parametric estimation yet, because not all data can be considered.\\
The detailed model is the last stage of the estimation and allocate the estimation with fifteen influence factors \cite{jenny}. These influence factors are divided into four categories.\\
\begin{enumerate}
	\item \textbf{Product Attributes}
	\begin{enumerate}
		\item \textit{Required Software Reliability}
		\item \textit{Size of the Application Database}
		\item \textit{Complexity of the Product}
	\end{enumerate}
	\item \textbf{Hardware Attributes}
	\begin{enumerate}
		\item \textit{Run-time Performance Constraints}
		\item \textit{Memory Constraints}
		\item \textit{Volatility of the Virtual Machine Environment }
		\item \textit{Required Turnabout Times}
	\end{enumerate}
	\item \textbf{Personal Attributes}
	\begin{enumerate}
		\item \textit{Influences Analyst Capability}
		\item \textit{Software Engineering Capability}
		\item \textit{Applications Experience}
		\item \textit{Virtual Machine Experience}
		\item \textit{Programming Language Experience}
	\end{enumerate}
	\item \textbf{Project Attributes}
	\begin{enumerate}
		\item \textit{Use of Software Tool}
		\item \textit{Application of Software Engineering Methods}
		\item \textit{Required Development Schedule}
	\end{enumerate}
\end{enumerate}

\subsubsection{Calculation of Implementation Time and Effort}

Because of the differences in each model of the COCOMO estimation, each model has its own equation for cost calculation. Each formula calculates the estimated time of the project in person-month (PM), which are stated by Boehm as 152 working hours resulting in one PM, with 19 working days per month and eight hour days \cite{boehm}. \\
The formula for the basic model calculates the complexity of a project with the KDSI value and the calculation factor of the project which gives the calculated PM as a result. This formula is as follows:\\
\begin{equation}
\textit{PM} = \textit{Complexity} \cdot (\textit{KDSI})^{\textit{Calculation Factor}} \label{cocomo:basic}
\end{equation}\\
The \textit{KDSI} value is the expected amount of code lines in the project. As described in table \ref{tab:projectclasses}, the Calculation Factor is derived from the KDSI value. To figure out the complexity multiplicand table \ref{cocomo:basicComplexity} describes for each project the suitable solution.\\
The equation for the intermediate model is the same as for the basic model (\ref{cocomo:basic}) \cite{boehm}. The difference is the changed multiplicand for this model, as described in table \ref{cocomo:complexity}.\\
The detailed model is an extension of the intermediate model that adds effort multipliers for each phase of the project to determine the cost drivers impact on each step. The effort is calculated as function of program size and a set of cost drivers given according to each phase of software life cycle \cite{boehm}. For an equation it is necessary to multiply all cost drivers \textbf{\(C_i\)} together, as described in the following formula:\\
\begin{equation}
C = \prod \limits_{i=1}^k C_i \label{cocomo:detailedcostdrivers}
\end{equation}\\
In this formula is \textit{k} the sum of all influence factors with 15 and each $C_i$ represents one of the 15 influence factors. This results in a combined influence factor which can be multiplied with KDSI value to get the total time for development (TDEV). But therefore the KDSI has to be calculated with the complexity factor for the detailed model from table \ref{cocomo:complexity}. This calculation can be described in the following equation \cite{boehm}:\\
\begin{equation}
TDEV = C \cdot (KDSI)^{Calculation Factor} \label{cocomo:detailed}
\end{equation}\\

\begin{table}[h]
	\centering 
	\setlength{\tabcolsep}{4pt}
	\begin{tabular}{|l|c|c|c|}\hline
		Complexity	&  Basic Model 		&  Intermediate Model	&  Detailed Model\\ \hline
		Small Project   	& 2.4      	& 3.2  					& 0.38	\\ \hline
		Medium Project 		& 3.0      	& 3.0  					& 0.35	\\ \hline
		Complex Project 	& 3.6 		& 2.8					& 0.43\\ \hline
	\end{tabular} 
	\caption{COCOMO complexity multiplicands} 
	\label{cocomo:complexity} 
\end{table}

\subsection{Comparison}

The Function Point analysis is useful for software projects of all size, mainly desktop based platforms. This is one of the biggest contrapositions as this technique is not good adaptable for console programs. Instead COCOMO is used for large corporate and government projects, including embedded firmware projects. The major source of criticism at COCOMO estimations is that they are inappropriate for small projects and that it is based on the waterfall model. It is recommended for large and lesser known projects to use COCOMO 2 instead.\\
The similarities between these models are that they both rely on algorithmic methods, with the difference that COCOMO is a logarithmic type and function point is linear. Both were created in the same time, where the project development cycle relied on linear procedure models and the used programming languages where procedural.\\
None of the techniques is necessarily better or worse than the other, in fact, their strengths and weaknesses are often complimentary to each other. There is no estimation technique that is the 'best fitting' for a project \cite{estimationanalysis}.

\section{Software for Cost Estimations - State of the art}
\label{sec:stateofart}

The big uncertainty of cost estimations is an actual problem for cost estimations. Stake holders and project managers don't trust the estimation output and calculate a surcharge on all estimations. In big projects this uncertainty factor is from 5\% up to 50\% is added on the estimated costs for a project \cite{fischer}. That is because many large projects are stopped due to budget overrun or incomplete requirements \cite{chaos}, which results in insecurities on the cost estimation output.\\  
Whereas the cost estimation is possible as paper work, there are several software products that allow it.\\
As there are many start-up IT companies founded in the last five years, there is no start-up company that develops cost estimation software \cite{dsm}. Also, there is no approach in developing a mobile application for cost estimations. Also the use of smartphones increased from 2013 to 2014 by 21\% and in the business environment 80\% use a smartphone \cite{faszinationmobile}. This opens up good opportunities on the market for the application.\\
There are three cost estimation tools that are mostly referred to.\\

\subsection{SEER - Cost Estimation Software}

The 'Seer - Cost Estimation Software' is developed by Galorath Inc. in Los Angeles, USA. Galorath Inc. offers consulting and software for cost estimation, decision support and project management. The company was founded in 1979 with the goal to improve the software and hardware development process in the industry. The next step was to improve their consulting quality by developing their own tools for this process. \textit{Seer} is the name of their tool set whit a large variety of different tools that support the development process of new products.
\\
Seer for Software is an estimation application for 'estimating, planning, analyzing and managing complex software projects'\footnote{\url{galorath.com/products/software/SEER-Software-Cost-Estimation}}. Based on the SEER design principles the software contains an annotated and guided interface for defining projects, a parametric simulation engine and numerous standard and custom reporting options. Due to an open architecture API the SEER application can be integrated with enterprise applications and departmental productivity solutions. Galorath specifies that all estimations within this software are repeatable and consistent.
\\
According to the software description, 'a high-level software estimate can be developed in a matter of minutes using SEER´s intuitive, window-based interface' \cite{pricesystems}. A dialog guides the user through the process of creating a new estimation. In the first screen the user has to set project name and decides whether he creates an empty project or starts with a scenario. The scenarios are example projects with some data for starting the estimation. In the next step the user chooses the estimation method he wants to use. The estimation methods are subdivided in functional, lines and sizing scale. Another feature of the software is the documentation and export function. All estimations can be exported to Microsoft Project, Microsoft Office, IBM Rational or other third-party software. SEER delivers a huge variety of estimation methods, guided processes and many possibilities for documentation and reporting of all estimations.
\\
The software can be bought in an estimator, project manager and studio version. The estimator version inherits only a standard estimation. The project manager and studio version allow estimation checking and access to the projects database. The studio version allows also independent crosscheck and verification. The price of each software version is nowhere specified on the homepage and must be requested.

\subsection{PRICE - Cost Estimation Software}

PRICE Systems L.L.C. provides agile estimating solutions. With their head office in Maunt Laurel, USA, and 12 locations worldwide they offer since 1969 estimating acquisitions\footnote{\url{http://www.pricesystems.com/}}. Their Software Development Cost Model is one of the oldest and most widely used software parametric models for software development projects. In 2003 PRICE released \textit{TruePlanning}, which contains methods that estimate the scope, cost, effort and schedule for software projects.
\\
For cost estimation, purchasing efficiency and budget planning PRICE Systems develops 'multi-faceted cost estimating solutions' \cite{pricesystems}. Their biggest project is the PRICE Estimating Systems (ESI) Framework that delivers solutions for estimating projects and cost management. This framework is not specialized for estimating only software projects but also other projects.
\\
The homepage describes that it is possible to estimate projects with all current state of the art cost estimations. The collaboration with other users is also possible as well as sharing the results through the program. This allows faster teamwork and a better estimation output. A data driven method for all estimations allows to compare the estimation with already completed projects. \textit{TruePlanning} allows also estimation output to a specific work breakdown structure or cost element structure for accurate top-down and bottom-up estimate comparisons. Mappings can be stored, retrieved, and modified to keep pace even as program activities change. Beyond specific cost estimating products and services, PRICE Systems offers strategic cost management services. They collaborate with estimators, engineers, project managers, and financial and executive management to design and implement integrated cost management systems that meet the unique challenges in the environment.
\\
Any details about the cost of the software or licensing have to be requested at \textit{Price Systems}.

\subsection{SLIM Estimate}

SLIM Estimate is a cost estimating software developed by Quantity Software Management\footnote{\url{http://qsma.com/slim-estimate}} (QSM), an estimation company with their head office in McLean, USA. QSM was founded in 1978 as a software management consultant company for measurement, estimation and controlling projects. Beside their consulting business segment they develop the SLIM Software which contains software for controlling, metrics and estimation.
\\
The SLIM Estimate software provides a flexible cost estimation which align the estimation to the project size. Integrated schedules offer simple exports of an existing calculation. The software can suggest alternate solutions calculated on past projects.
\\
The homepage does not go into detail how this suggestion works. For a new estimation the user can use the SLIM Database and the QSMA database, which can be used for new estimations.
\\
For better use in the company the SLIM Estimate inherits interfaces to MS Office and the possibility export estimations as a web representation. SLIM Estimate delivers hereby a software package with many functionalities. Especially the access to a large database and the knowledge base of QSM is a big plus value to the application. As a result, it is possible to access the experience of past projects and get the benefit from this experience.


\subsection{Commonalities}

All described cost estimation tools have in common, that they inherit different estimation techniques. The user can always decide with which technique he wants to use. They are all tools that are on the market for many years and rely on experience from project managers. Each software has a different approach on the implemented estimation process and collaboration of project estimations. They all have a database with previous cost estimations but with extra costs for accessing it.\\
The User interface is like in many large software projects more functional and does not contain a modern design. None of them has an approach for developing a mobile application and rely on Microsoft Windows as their OS.

\subsection{Kinvey}

A special software developer is Kinvey from Boston whose main business is the development of mobile applications. They don't develop a certain cost estimation tool, but in terms of general cost estimations, the company is still very interesting\footnote{\url{http://www.kinvey.com/mbaas-savings-calculator}}.
\\
Kinvey offers an online estimation for mobile applications in which the potential customers can estimate the costs for their application. The estimations show the time the customer would have to spend for development and the costs for development at Kinvey. The cost of developing through Kinvey here are always cheaper than a proprietary development.
\\
The cost estimation process is simple. The estimation schedule is a one-page where the user can select the components of his application. The user is guided through some question about features the application should contain. After each step the user can see the actual cost estimation of the project.
\\
The costs are calculated from the selected components in man days. It is not possible to define and add your own components for the estimation. The individual items that the user can choose are sorted by groups and displayed graphically. Estimating the cost for a mobile application development can also be used for own estimations, but there is no way to export this estimation.

% !TeX spellcheck = en_US
\chapter{Application concept}

Lorem Ipsum
\chapter{Implementation}

The used libraries and selected implementation approaches are described in this chapter. \textit{Android Studio} was used as the \textit{IDE} for development and \textit{GitHub} was used for version management. 

\section{Used Libraries}

For the implementation were two external libraries necessary. These libraries are called \textit{POI} and \textit{mpandroidchartlibrary}. \textit{POI} is a library from the \textit{APACHE Software Foundation} and is open source. It allows a conversion of data into Microsoft documents. This library is used
in the \textit{Export} component and allows to create \texttt{xls} documents with the project data. The \textit{mpandroidchartlibrary} library is also an open source library developed by Philipp Jahoda. This library provides many different diagram types for Android such as pie and bar charts. This charts are used for the estimation technique analysis and for the project statistics. 

\section{Project Structure}

Android divides the project into an \textit{assets}, \textit{resources} and \textit{java} folder. The \textit{assets} folder stores project icons and the two databases of the project. After installation of the \texttt{apk}, these files are moved to the internal folder space of the application. The \textit{resources} folder contains every \texttt{layout} file, \texttt{string} resources and the android \texttt{icons}. All resource files ar typically \texttt{XML} files or in case of images \texttt{png} files. The \textit{java} folder contains the complete source code. This folder was divided into the sub folders \textit{Activities}, \textit{Fragments}, \textit{DataObjects}, \textit{Server} and \textit{Util}. The folders \textit{Activities} and \textit{Fragments} contain the associated classes for each layout and are responsible to control the user input. In the folder \textit{DataObject} are all data classes stored. These are objects that only store data and don't have any logical components. The \textit{Server} folder contains at the moment only the \texttt{ServerConnector} class. This folder was created to separate the future server connection. Last folder is the \textit{Util} folder. It contains classes that process data, for example the \texttt{DatabaseHelper} class.

\section{Pre-filled Database}\label{exDB}

The advantage of a pre-filled database is a shorter initialization after first start of the application. The database doesn't have to run many \textit{SQL} scripts to create tables and insert values. Also values and tables can be created with an external tool such as \textit{SqliteBrowser}\footnote{\url{http://sqlitebrowser.org/}}. This allows testing of the tables and \textit{queries} that are used in the application. To use this database two steps are essential:
\begin{enumerate}
	\item \textbf{Create Database}
	\item \textbf{Open Database}
\end{enumerate}
Both methods are inherited in the class \texttt{DataBaseHelper}. Listing \ref{createdb} shows the method \texttt{createDataBase()} which is responsible for the creation of the database. At first the method checks with \texttt{checkDataBase()} if the database does already exist in the \texttt{application} folder. If it does not exist the method calls \texttt{copyDatabase()} which will read the database from the \texttt{assets} folder and write a new database to the application folder.
\begin{lstlisting} [caption=Method \textit{createDataBase}, label=createdb] 
public void createDataBase() throws IOException
{
	boolean dbExist = checkDataBase();
	if (!dbExist)
	{
		this.getReadableDatabase();
		try
		{
			copyDataBase();
		} catch (IOException e)
		{
			Log.d("ERROR", "Database could not be copied " + e.getCause());
			throw new Error("Database could not be copied");
		}
	}
}
\end{lstlisting}
Open the database is done with the method \texttt{openDataBase()}, which is described in listing \ref{opendb}. The class variable \texttt{projectEstimationDataBase} is initialized here with the \texttt{SQLiteDatabase} command \texttt{openDatabase} which create a \texttt{SQLiteDatabase} object with the path where it is stored. This allows the execution of \texttt{SQL} queries in this object.
\begin{lstlisting} [caption=Method \textit{openDataBase}, label=opendb] 
public void openDataBase() throws SQLException
{
	String myPath = DB_PATH + DB_NAME;
	projectEstimationDataBase = SQLiteDatabase.openDatabase(myPath, null,
										 SQLiteDatabase.OPEN_READWRITE);
}
\end{lstlisting}
\section{The Database Helper}

The \texttt{DataBaseHelper} class has the most lines of code in the project. As \texttt{SQL} queries can be generalized it is not possible to generalize the preprocessing of the data. Different date needs different treatment for the calling method. Each table has also various column names what is another difficulty for generalization. \\
All requests with \texttt{SELECT} statements have a similar structure. As described in listing \ref{selectStatement}, the first step is to build a \texttt{String query} with the table name and parameter from the method. Next step is to get a readable database from the \texttt{SQLiteDatabase} object of this class. The query will be executed on this object and creates a \texttt{Cursor}, which is the result table. To get the values from this \texttt{Cursor} each column has to be addressed and saved into a variable which will be returned to the caller. If more values are needed those have to be packed into other data structures.
\begin{lstlisting} [caption=Method \textit{openDataBase}, label=selectStatement] 
public void selectSomethingFromTable(String id)
{
	String query = String.format("SELECT _id FROM <TABLENAME> WHERE id = '%s'",
											 id);
	SQLiteDatabase db = this.getReadableDatabase();
	String name = "";
	try (Cursor c = db.rawQuery(query, null))
	{
		if (c.moveToFirst())
		{
			name = c.getInt(c.getColumnIndex("<COLUMN-NAME>"));
		}
	}
	db.close();
	return id;
}
\end{lstlisting}
For the query types \texttt{DELETE} and \texttt{ALTER TABLE} a method needs to instantiate a writable database with \texttt{'SQLiteDatabase db = this.getWritableDatabase()'}. The \texttt{db} objects offers the possibility \texttt{update} to change existing values in a table. It gets an object \texttt{ContentValues} as a parameter which inherits the column name and value which has to be updated. The other to parameters for \texttt{update} are the table name and selection to find the right row. To delete a row the method \texttt{delete(String table, String whereClause, String[] whereArgs)} is needed. It has the parameters that indicate the table and which row should be deleted.

\section{Multilingual Strings}

As described in section \ref{accessingresources} all non user edited names in the database are connected with the android \texttt{string.xml} resource. To achieve this connection a \texttt{HashMap} is needed which connects the resource names with the \texttt{id} \textit{Android} assigns to each string. It is important that the name in the database is exactly the same as in the string \texttt{xml}, otherwise the string won't be found and causes an error. It is also important to assure that every string key is unique to avoid false string being loaded. This map is called \texttt{resourcesIdMap} and has the name from the database as key and the associated resource id as value. Two methods that process this \texttt{HashMap}: \texttt{getStringResourceValueByResourceName} \texttt{(String resource Name)} and \texttt{getStringResourceNameByResourceValue (String resourceValue)}. The parameter \texttt{ResourceName} is the name that is described in the database and \texttt{ResourceValue} is the string from \texttt{strings.xml}.\\
Listing \ref{resourcebyName} shows the source code for reading the value of a resource with it's key. For example the key \texttt{'test\_string} returns as a result the string \texttt{'test'}.
\begin{lstlisting} [caption=\textit{getStringResourceValueByResourceName}, label=resourcebyName] 
public String getStringResourceValueByResourceName(String resourceName)
{
	int resID = context.getResources().getIdentifier(resourceName, "string", 
	context.getPackageName());
	String name = context.getResources().getString(resID);
	DataBaseHelper.resourcesIdMap.put(name, resID);
	return name;
}
\end{lstlisting}

\section{Load Project Icons}\label{impl:loadProjectIcons}

In the table \texttt{ProjectIcons} are the informations for all available icons stored. The important column for loading the icon is called \texttt{path}. It stores the relative path to the icon and the icon name. The method \texttt{loadProjectIcon} has to look if the icon exists at this position. As described in listing \ref{loadicon}, this icon is decoded in a \texttt{Bitmap} and returned to the caller. \texttt{Android} can present a \texttt{Bitmap} without further effort.

\begin{lstlisting} [caption=Method \textit{loadProjectIcon}, label=loadicon] 
public Bitmap loadProjectIcon(String path)
{
	AssetManager assetManager = this.context.getAssets();
	InputStream istr;
	Bitmap projectIcon = null;
	try
	{
		istr = assetManager.open(path);
		projectIcon = BitmapFactory.decodeStream(istr);
	} catch (IOException e)
	{
	}
	return projectIcon;
}
\end{lstlisting}
\section{Calculate Person Days}

This functionality is at the moment only adapted to the \textit{Function Point} technique. For calculating the \textit{person days} or \textit{man days} two methods are necessary. One is the \texttt{evaluateFunctionPointPersonDaysWithBaseProductivity}, when there are no terminated projects to calculate a \textit{points-per-day} value. \texttt{evaluateFunction- PointPersonDaysWithExistingProductivity} is called when there are already terminated projects which will help get the best fitting \textit{points-per-day} value. The calculation with the base productivity checks the database for the range of the total points of the project. If a project hast \textit{total points} of \textit{500}, for example, it fits in the base productivity that goes from \textit{350} to \textit{650} points, which has a \textit{points-per-day} value of \textit{16}. These \textit{500} points will then be divided with \textit{16} which results in \textit{31.25} \textit{man days}.\\
If there are enough terminated project the algorithm tries to calculate a more accurate \textit{points-per-day} value. The calculation checks all terminated projects for a project with less total points and for a project with more total points than the calculated project. Within these values the algorithm searches for those projects which total points are nearest to the calculated project. If one of these items are empty the algorithm selects the base productivity for this value instead. Listing \ref{evaldays} shows the code that is executed afterwards. For the smaller and bigger item the average \textit{points-per-day} are calculated. This value is then divided with the selected projects \textit{evaluated points} which returns the \textit{evaluated days} for the project.
\begin{lstlisting} [caption=Evaluate Days, label=evaldays] 
double averagePointsPerDay = (smallerItem.getPointsPerDay() + 
				biggerItem.getPointsPerDay()) / 2;
averagePointsPerDay = roundDoubleTwoDecimals(averagePointsPerDay);
evaluatedDays = roundDoubleTwoDecimals(project.getEvaluatedPoints() 
				/ averagePointsPerDay);
\end{lstlisting}
\section{Find related Projects}

The relation of two projects is calculated in the \texttt{ProjectRelationSolver} class. It is initialized with all projects that are listed in the database which inherits both active and deleted projects. Also the selected project for comparison is a parameter at initialization. The main method for finding related projects is \texttt{getRelatedProjects} and has the \texttt{relationBorder} as parameter which defines the percentage border for relation. It calls for each project the method \texttt{compareWithChosenProject} which compares a project with the selected one. As most of the property distances are save in the database, this method only needs to select the right distance from the database. Listing \ref{selectdistance} shows this selection with the development market distance as example. It calls the method \texttt{getPropertyDistance} from the \texttt{DatabaseHelper} and is a generalized method. It gets the table name of the property and the table name of the distance as parameter. Also the names of the columns for selecting the right value. The last two parameters are the values that have to be compared.
\begin{lstlisting} [caption=Development Market Distance, label=selectdistance] 
double developmentMarketDistance = databaseHelper.getPropertyDistance
			("DevelopmentMarkets", "DevelopmentMarketComparison", 
			"market_id_1", "market_id_2", "comparison_distance",
			project.getProjectProperties().getMarket(),
		    p.getProjectProperties().getMarket());

\end{lstlisting}
For \texttt{software architecture} and \texttt{programming language} these distance has to be calculated. This is done by loading the properties from the database and subtracting them. These distances are then added to the \texttt{distanceSum} which is then calculated with the equation in listing \ref{calcdistance}. For logging purposes the calculation was separated into two parts. Before returning the \texttt{differencePercentage} has to be subtracted from \textit{100} to get the relation.
\begin{lstlisting} [caption=Percentage Difference Calculation, label=calcdistance] 
differencePercentage = databaseHelper.roundDoubleTwoDecimals(distanceSum 
										/ 103) * 100;
\end{lstlisting}
\section{ListView Elements and ViewHolder}

The difficulty with \texttt{ListViews} was to achieve that each item remember his content and can be processed. For example selecting a \textit{button} in the list element which returns the name of the item. This was achieved by using so-called \texttt{ViewHolder}. These are just sub-classes that inherit all \texttt{View} elements of the list item. This is done in the adapter of \texttt{ListViews}. The method \texttt{getView} is called for filling the \texttt{ListView} with items. One parameter of this method is the \texttt{convertView} which represents the actual item. The trick to achieve that each element remembers its content to add the \texttt{ViewHolder} as a tag to the \texttt{convertView}. Listing \ref{lvholder} shows an example of this allocation. For each list item a new \texttt{ViewHolder} is initialized. Each including \texttt{View} element has to be assigned with the element inside the \texttt{convertView}. Finally the \texttt{holder} is assigned as a tag to the \texttt{convertView}.
\begin{lstlisting} [caption=ListView Holder example, label=lvholder] 
convertView = vi.inflate(R.layout.
	function_point_influence_factorset_with_subitems_list_item, null);
holder = new ViewHolder();
holder.tvShowSubitems = (TextView) convertView.findViewById(
	R.id.tvShowSubitems);
convertView.setTag(holder);
\end{lstlisting}

\chapter{Software Test}

Lorem Ipsum
\chapter{Conclusion}

This paper presented the possibility to process \textit{cost estimation} with a mobile application. It tries to give an approach to make projects \textit{comparable} and build a base for a mobile cost estimation application. It focuses on the implementation of \textit{Function Point} and makes a complete estimation possible.\\
The developed application allows a comparison between projects. It relies on \textit{seven} different project properties. These are calculated with a distance which gives the \textit{total difference} of two projects. \\
To use the application properly it is assumed that the user knows how the estimation techniques work. He has the possibility to use the help inside the application but a basic understanding is required.\\
One of the biggest problems was to create a user interface which inherits all functions to estimate a project and also supports the android design principles.\\
Another problem was to create a logical algorithm for comparing projects. The biggest problem was to find properties which can be compared and can be set for every project. These properties must also have relevance for the algorithm.\\
As an outlook for further development are many features and additions possible. One addition is to implement more estimation techniques such as \textit{COCOMO}. This would support more informations about how long a project can take. Because different technique have a different effort as output. Another addition is to create a server which stores all estimations from the user. This would support more possibilities to improve the estimations. All users with access can rely on the informations of other users. It could also allow sharing of the projects. A server management tool would provide the possibility to manage the master data of the project comparison from the server. Adaption of the application for tablets would provide the possibility for more functionalities and different user experience. Also the comparison functionality has to be refined and it must be evaluated if more project properties are necessary to get more exact results.\\
All in all it can be said, that the mobile application for cost estimations offers a promising alternative to the traditional estimation software, that is worth further research and evaluation.

% ...
%--------------------------------------------------------------------------
\backmatter                        		% Anhang
%-------------------------------------------------------------------------
\bibliographystyle{geralpha}			% Literaturverzeichnis
\bibliography{literatur}     			% BibTeX-File literatur.bib
%--------------------------------------------------------------------------
\printindex 							% Index (optional)
%--------------------------------------------------------------------------
\begin{appendix}						% Anhänge sind i.d.R. optional
   %\include{chapters/Glossar}			% Glossar  
   \chapter{Requirements}
\label{app:RE}

This section contains all requirements for the developed application. It limits on the \textit{description}, \textit{title} and \textit{id} of the requirements to save space.
 
   \chapter{Function Group}
\label{app:funcgroup}

This section inherits an overview of all function groups of the application \textit{MobileEstimate}.
\begin{figure}[h] 
	\centering 
	\includegraphics[width=10cm]{images/ScreenOverview.pdf} 
	\caption{- Function Group Part 1} 
	\label{completefunctiongroup}
\end{figure}
\begin{figure}[h] 
	\centering 
	\includegraphics[width=13cm]{images/ScreenOverview2.pdf} 
	\caption{- Function Group Part 2} 
	\label{completefunctiongroup2}
\end{figure}

   \chapter{Property Distances}
\label{app:propdist}

This section introduces the distances with which each property was calculated. This is limited to the calculated distances as they are stored in the database. The calculation itself is described in section \ref{ProjectProperties}.
 
%   \chapter{Entity Relation Diagram}
\label{app:erdia}

The complete \textit{entity relation diagram} of the \textit{project database} is shown in this section.
 
   \chapter{Test Cases}
\label{app:testcases}

This section contains all \textit{Test Cases} with which the application was tested.
  
   \include{chapters/Selbststaendigkeitserklaerung}	% Selbstständigkeitserklärung
\end{appendix}

\end{document}
